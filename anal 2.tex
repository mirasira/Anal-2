\documentclass[12pt, letterpaper]{article}
\usepackage[utf8]{inputenc}

\title{ANAL 2}
\author{Miroslav D.}

\usepackage{pgfplots}
\usepackage{amsmath}
\usepackage{amssymb}
\usepackage{natbib}
\usepackage{graphicx}
\usepackage{hyperref}
\graphicspath{ {images/} }


\begin{document}
\maketitle
Tento dokumetn shrnuje látku potřebnou ke zkoužce z MA2 v letech 20/21. Zkoužka se skládá z 5-ti příkladů.


% příklad 1
\section*{Příklad č.1.}

Příklad na aplikaci derivací.
\begin{itemize}
  	\item Tečná rovina v bodě na nějakou plochu.
	\item Největší/nejmenší funkční hodnota na nějakém průsečíku ploch.
	\item Extrémy nějaké plochy.
	\item Nějaká "slovní úloha" - největší oběm/plocha nějakého útvaru/tělesa.
\end{itemize}

\subsection*{Tečná rovina}
\textbf{Zadání:} elipsoid: \[x^2 + 2y^2 + 3z^2 = 1\] rovina: \[3x - y + 3z  -1 = 0\]
Najít body na elipsoidu rovnoběžné s rovinou

Ze zadané funkce potřebujeme zjistit gradient - ten nám určuje "spád" tečné roviny v libovolném bodě funkce - podobně jako derivace z analýzy 1. Ten se určuje derivací funkce podle x, y, z.
\[ \frac{\partial F}{\partial x} = 2x \]
\[ \frac{\partial F}{\partial y} = 4y \]
\[ \frac{\partial F}{\partial z} = 6z \]

Dále ze zadané roviny zjistíme normálový vektor - to je jednoduché, rovnice roviny je pomocí normálového vektoru popsána.
\[\vec{n} = (3, -1, 3)\]

Teď zjistíme kdy se gradient a normálový vektor + lambda rovnají a dosadíme do rovnice plochy
\[\lambda (3, -1, 3) = (2x, 4y, 6z)\]
\[x=\frac{3}{2}\lambda \]
\[y=-\frac{1}{4}\lambda \]
\[z=\frac{1}{2}\lambda \]

Dosadíme do rovnice plochy 

\[\left(\frac{3}{2}\lambda \right)^2 + 2\left(-\frac{1}{4}\lambda \right)^2 + 3\left(\frac{1}{2}\lambda \right)^2 = 1\]

Spočítáme

\[\lambda = \pm \frac{2\sqrt{2}}{5} \]

dosadíme zpět a máme výsledek

\[A_1 = \left(\frac{3\sqrt{2}}{5},  -\frac{\sqrt{2}}{10}, \frac{\sqrt{2}}{5}\right), A_2 = \left(-\frac{3\sqrt{2}}{5},  \frac{\sqrt{2}}{10}, -\frac{\sqrt{2}}{5}\right) \]

\subsection*{Extrémy plochy}
Je to podobné jako extrémy v MA1, ale tady hledáme plochy rovnoběžné s plochou xy.

Lokální extrémy funkce $z = 4x - 3x^3 - 2xy^2$
Zderivujeme funkci
\[\begin{aligned}
\frac{\partial F}{\partial x} &= 4 - 9x^2 - 2y^2 = 0 \\
\frac{\partial F}{\partial y} &= -4xy = 0
\end{aligned}\]

Z druhé rovnice x = 0 a y = 0, dosadíme do první, vyjde nám $(0, \pm \sqrt{2})$ a $(\pm \frac{2}{3}, 0)$
Teď musíme zjistit, jestli se jedná o maximum/minimum, nebo jestli to je vůbec extrém, k tomu ná poslouží Hessova matice (asi ne podle toho nácka xd).
Je to symetrická matice.

Tohle je pro 3 x 3 matici 
\[\begin{pmatrix}
\frac{\partial F}{\partial x^2} & \frac{\partial F}{\partial x \partial y} & \frac{\partial F}{\partial x \partial z}\\
\frac{\partial F}{\partial y \partial x} & \frac{\partial F}{\partial y^2} & \frac{\partial F}{\partial y \partial z}\\
\frac{\partial F}{\partial z \partial x} & \frac{\partial F}{\partial z \partial y} & \frac{\partial F}{\partial z^2}
\end{pmatrix}\]

Tohle je pro 2 x 2 matici 

\[\begin{pmatrix}
\frac{\partial F}{\partial x^2} & \frac{\partial F}{\partial x \partial y}\\
\frac{\partial F}{\partial y \partial x} & \frac{\partial F}{\partial y^2} 
\end{pmatrix}\]


Po dosezení dostáváme
\[\begin{pmatrix}
-18x & -4y\\
-4y & -4x 
\end{pmatrix}
\approx 
\begin{pmatrix}
-9x & -2y\\
-2y & -2x
\end{pmatrix}\]

Teď se dosazujou hodnoty těch extrémů co vyšly před tim a počítaj se determinanty - nejdřív $1\times1$ matice (ta nahoře vlevo), potom $2\times2$ (potom$3\times3$).
D1 je determinant matice $1\times1$, D2 je detrminant matice$2\times2$.
\[(0, -\sqrt{2}) \to
\begin{vmatrix}
0 & -2\sqrt{2}\\
-2\sqrt{2} & 0 
\end{vmatrix} \to
D_1 = 0, D_2 = - 8\]

\[(0, \sqrt{2}) \to
\begin{vmatrix}
0 & 2\sqrt{2}\\
2\sqrt{2} & 0 
\end{vmatrix} \to
D_1 = 0, D_2 = - 8\]

\[\left(-\frac{2}{3}, 0\right) \to
\begin{vmatrix}
-6 & 0\\
0 & -\frac{4}{3} 
\end{vmatrix} \to
D_1 = -6, D_2 = 8\]

\[\left(\frac{2}{3}, 0\right) \to
\begin{vmatrix}
6 & 0\\
0 & \frac{4}{3} 
\end{vmatrix} \to
D_1 = 6, D_2 = 8\]


Teď jenom stačí rozlišit kdy je extrém - fancy se tomu řiká sylvestrovo kritérum. Když jsou všechny determinanty kladný - jedná se o lokální minimum, 
když se znaménka determinantů střídají a první determinant je záporný, jedná se o lokální extrém.
Když determinant není nulový a neplatí žádná s předchozích, jedná se o sedlový bod (není to minimum ani maximum).
Když je nějaký diskriminant nula, tak sylvestrovo kritérium nemůže rozhodnout a musíme extrém určit jinak (např. po křivkách).
Maximum je bod $(\frac{2}{3},0)$, minimum je bod $(-\frac{2}{3},0)$

\subsection*{Největší/nejmenší funkční hodnota}
V zadání jsou dvě funkce, f ve tvaru $f(x_1, ......, x_n)$ a $g(x_1, ......, x_n) = a$
Sestaví se lagrangeova funkce $L =f(x_1, ......, x_n)  + \lambda g(x_1, ......, x_n - a)$ a řeší se to dost podobně jako extrémy.

Příklados:
Zjistěte nejmenší a největší hodnotu funkce  $f(x,y) = xy$ v bodech elipsy $x^2 + 2y^2 = 1$. Sestavíme lagreangovu funkci.

\[L = xy = \lambda(x^2 + 2y^2 - 1)\]

\[\begin{aligned}
\frac{\partial L}{\partial x} & = y = 2\lambda x\\
\frac{\partial L}{\partial y} & = x = 4\lambda y\\
\frac{\partial L}{\partial \lambda} &= x^2 + 2y^2 -1 = 0
\end{aligned}\]

Vyřešíme - $(\pm \frac{1}{\sqrt{2}}, \frac{1}{2} )$, $(\pm \frac{1}{\sqrt{2}}, -\frac{1}{2})$, Nejvyšší a nejnižší hodnotu získáme dosazenim.

maximum - $\pm(\frac{1}{\sqrt{2}}, \frac{1}{2} )$,  minimum - $\pm( - \frac{1}{\sqrt{2}}, \frac{1}{2} )$

\subsection*{Slovní úloha}
Je to prakticky úloha na sestavení lagreangovy funkce, je víc přemýšlicí.
\newline


Příklad: $54\pi cm^3$ - objem plechovky, dno a víko je z materiálu za 0.25 Kč/$cm^3$, plášť z materiálu za 0.5 Kč/$cm^3$. Nalezněte rozměry, tak aby byla cena minimální.

Sestavíme lagreangovu funkci.
\[L = 0.25\cdot 2 \pi r^2 + 2\pi r h 0.5 - \lambda(\pi r^2 h - 54 \pi)\]
$2 \pi r^2$ je plocha víka a dna, $ 2 \pi r $ je plocha pláště, oběm plechovky je $\pi r^2 h = 54 \pi$
Zderivujeme

\[\begin{aligned}
\frac{\partial L}{\partial x} & = r + h -2 \lambda r = 0\\
\frac{\partial L}{\partial y} & = r - \lambda r^2= 0\\
\frac{\partial L}{\partial \lambda} &= r^2 h = 54
\end{aligned}\]

Spočítáme, výsledek je h = r = $\sqrt[3]{54}$ = $3\sqrt[3]{2}$

% příklad 2

\section*{Příklad č.2.}

\subsection*{Prohození pořadí integrace}

Nejlepší je si nakreslit obrázek - funkce nejsou tak složité aby to nešlo, v nouzi se dá použít kalkulačka - náčrtek nemusí být tak přesný, čtyři funkční hodnoty pro slozitěší funkce by na to měly stačit.

Příklad:
\[\int_1^2\int_1^{\sqrt{5-x^2}}  f dy dx\]

\begin{figure}[h]
	\centering
	\includegraphics[width=1\textwidth]{geogebr}
\end{figure}

\subsection*{Přehození integrálu}

integrál si můžeme představit jako \[\int_{x = 1}^{x = 2}\int_{y = 1}^{y = \sqrt{5-x^2}}  f dy dx\]
Je vidět, že chtěná plocha je v útvaru mezi body A, B, C.
Teď potřebujeme integraci přehodit tak, aby vnější integrál byl podle y a vnitřní podle x.
Funkce je ohraničená podle y = 2 a y = 1 - vnější integrál bude vypadat stejně.
X je tedy ohraničeno podle x = 1 a oblouku - vyjádříme ho pomocí x. \[x = \sqrt{5-y^2}\]
Tady pozor na + nebo -, z obrázku se to dá poznat jednoduše, podle toho v jakém kvadrantu je útvar.
Napíšeme výsledek

\[\int_1^2\int_1^{\sqrt{5-y^2}}  f dy dx\]

\subsection*{Polární souřadnice}

Zase nám pomůže obrázek. $\varphi$ je úhel odkud kam budeme integrovat - z obrázku vidíme, že to jsou body C a B - spočítáme si úhel. $\tan \varphi_1 = 2$, $\tan \varphi_2 = \frac{1}{2}$.
Je tady ale malej prank, když přecházíme přes bod A, tak se nám změní dolní hranice pro $\rho$, proto musíme integrál rozdělit na dva. Funkce si vyjádříme pomocí polárních souřadnic.

\[\int_{\frac{\pi}{4}}^{arctg 2}   \int_{\frac{1}{\cos{\varphi}}}^{\sqrt{5}}  f    d\rho d\varphi +  \int_{arctg \frac{1}{2}}^{\frac{\pi}{4}} \int_{\frac{1}{\sin{\varphi}}}^{\sqrt{5}}  f d\rho d\varphi   \]

% příklad 3

\section*{Příklad č.3.}

Aplikace různých typů integrálů

\begin{itemize}
	\item Gaussova věta
  	\item Greenova věta
	\item Stokesova věta
	\item Těžiště tělesa
	\item Moment hybnosti tělesa
\end{itemize}



\subsection*{Gaussova věta}
Pomocí Gaussovy věty se počítá tok pole nějakým tělesem.
Potřebujeme zjistit divergenci vektorového pole(vynásobit skálárně nabla operátorem), potom "jenom" spočítat objemový integrál toho pole.
Nezapomenout na jakobiány při převádění na jiné souřadnicové systémy.
Pro cylindrickou a polární je jakobián $\rho$.
Pro sférickou $\rho^2 \sin \theta$


Příklad: pomocí Gaussovy věty spočítejte tok pole $\vec{F}$ orientovanou plochou M. $\vec{F} = (0, 0, \frac{1}{3}z^3)$, M je sféra $x^2 + y^2 + z^2 = r^2$.

\[\nabla \vec{F} = \frac{\partial \vec{F_1}}{\partial x} + \frac{\partial \vec{F_2}}{\partial y} + \frac{\partial \vec{F_3}}{\partial z} = z^2\]

Protože se jedná o sféru, převedeme si souřadnice na sférické.
\[x = \rho \sin \theta \cos \varphi \]
\[y = \rho \sin \theta \sin \varphi \]
\[z = \rho \cos \theta \]

\[z^2 = \rho^2 \cos^2 \theta \]

Nakonec sestavíme trojný integrál. Dobré je si představit jak se ty meze vytváří.
Nejdříve opíšeme kružnici pomocí $\theta$, to je od $0$ do $2\pi$.
Dál vyplníme celou tuto kružnici od počátku do poloměru, $\rho$ je od $0$ do $r$.
A nakonec tu krožnici rotujeme - stačí nám ale jen $180^{\circ}$ na vyplnění celé koule, takže $\theta$ je od $0$ do $\pi$.
Nezapomenout vynásobit jakobiánem. Spočítáme.
\[\int_0^\pi\int_0^r\int_0^{2\pi}  \rho^4 \cos^2 \theta \sin \theta d\varphi d\rho d\varphi = \frac{4 \pi}{15}r^5\]  


\subsection*{Greenova věta}
Greenova věta je Stokesova věta ve 2D.
Greenova věta nám převádí křivkový integrál 2. druhu po uzavřené křivce na dvojný integrál.

\[ \oint_c (L dx + M dy) = \iint_D \left(\frac{\partial M}{\partial x} - \frac{\partial L}{\partial y} \right) dx dy\]

Příklad: Pomocí Greenovy věty spočítejte jaká práce se vykoná při pohybu bodu o hmotnosti m = 1 po obvodu trjúhelníku s vrcholy (0,0), (3,0) a (3,3). Pohyb se děje v kladném smyslu v poli $\vec{F} = (y^2 - x^3, x^2 - y^3)$

\[\begin{pmatrix}
\frac{\partial F}{\partial x} & \frac{\partial F}{\partial y}\\
y^2 - x^3 & x^2 - y^3
\end{pmatrix} = 2x - 2y\]

Sestavíme integrál a spočítáme.

\[\int_0^3 \int_0^x 2x - 2y  dy dx = 9\]

\subsection*{Stokesova věta}
Stokesova věta je Greenova věta ve 3D.
\subsection*{Těžiště tělesa}
Těžiště se určuje pomocí vzorců
\[x_t = \frac{\iint_M x_\rho}{\iint_M \rho}, \qquad y_t = \frac{\iint_M y_\rho}{\iint_M \rho}\]

Příklad: Najděte těžiště kužele P = $ \{ (x, y, z) \in  \mathbb{R}^3 | \sqrt{x^2 + y^2} \leq z \leq h \} $ je-li hustota v bodě (x, y, z) rovna jeho vzdálenosti od roviny xy.

Protože kužel symetrické těleso, můžeme z-tovou složku zanedbat. Potřebujeme spočítat tři dvojné integrály.

\subsection*{Moment hybnosti tělesa}
% příklad 4

\section*{Příklad č.4.}

Vyšetřování potenciálních polí

Prvním krokem, je vždy zjistit, jestli (nebo kdy) je pole potenciální - zrotujeme pole (vektorově ho vynásobíme s nabla operátorem), když nám vyjde 0, tak je pole potenciální (případně zvolíme hodnotu tak aby potenciální bylo).

Zjistěte, které z polí $\vec{F} = (y + z, x + z, x + y)$ a $\vec{G} = (x, yz^2, 3)$ je potenciální a nalezněte jeho potenciál.


\[\vec{\nabla} \times \vec{F} = \begin{vmatrix} e_1 & e_2 & e_3 \\
\frac{\partial F}{\partial x} & \frac{\partial F}{\partial y} & \frac{\partial F}{\partial z} \\ 
y + z & x + z &  x + y\end{vmatrix} = (1-1)e_1 + (1-1)e_2 + (1-1)e_3 = 0\]

\[\vec{\nabla} \times \vec{G} = \begin{vmatrix} e_1 & e_2 & e_3 \\
\frac{\partial F}{\partial x} & \frac{\partial F}{\partial y} & \frac{\partial F}{\partial z} \\ 
y & yz^2 &  3\end{vmatrix} = (0-2z)e_1 + (0-0)e_2 + (0-1)e_3 = (-2z, 0, -1)\]

Potenciální je jen pole F.

Spočítání potenciálu. Složky pole se rovnají parciální derivaci.

\[(1) \quad \frac{\partial F}{\partial x} = y + z \]
\[(2) \quad  \frac{\partial F}{\partial y} = x + z \]
\[(3) \quad  \frac{\partial F}{\partial z} = x + y \]


\hypersetup{
    colorlinks=true,
    linkcolor=blue,
    filecolor=red,      
    urlcolor=cyan,
    }

Je to celkem jednoduchý algoritmus, je vypsaný pod textem, pokusim se ho nějak popsat, nejlepší je na to asi \href{https://www.youtube.com/watch?v=vSN8DivqVpQ&list=PLQL6z4JeTTQne5N_yD6X-hK3FW57rAb5b&index=19}{video od Doc. Bohaty}. Vezmeme první řádek, který zintegrujem, tím zjistíme kolik se rovná první řádek.
Potřebujeme zjistit kolik je $C_1$, tak zderivujeme první řádek podle y - všechny proměnné z pravé strany by se měli vykrátit - nenarazil jsem ještě na příklad kde by to tak nebylo, je možný, že je to tim, že je to pole potenciální.
Zjistíme integrací kolik je $C_1$ a dosadíme to do prvního řádku, takhle to vlastně opakujeme až do konce.

$$
\begin{aligned}
1. \quad F(x, y, z) &= xy + xz + c_1(yz) \\
2. \quad x + \frac{\partial c_1}{\partial y} &= x + y\\
3. \quad \frac{\partial c_1}{\partial y} &= z\\
4. \quad c_1 &= yz + c_2(z)\\
5. \quad F(x, y, z) &= xy + xz +  yz + c_2(z) \\
6. \quad x + y \frac{\partial c_2}{\partial z} &= x + y\\
7. \quad c_2 &= 0 + c\\
8. \quad F(x, y, z) &= xy + xz +  yz + c
\end{aligned}
$$
Toto je výsledný potenciál








% příklad 5

\section*{Příklad č.5}

Fourierova nebo Taylorova řada

\subsection*{Taylorova řada}
Taylorova řada převádí funkce na nějaký nekonečný součet.
Důležité vzorce - bývají tam napsány a dají se odvodit, přesto bych si je radši zapamatoval.

\[\begin{aligned}
\frac{1}{x+1} &= \sum_{n=0}^{\infty} x^n, \quad x \in \langle -1, 1 )\\
e^x &= \sum_{n=0}^{\infty} \frac{x^n}{x!}, \quad x \in \mathbb{R} \\
\sin x &= \sum_{n=0}^{\infty} (-1)^n \frac{x^{2n+1}}{(2n+1)!}, \quad x \in \mathbb{R}\\
cos x &= \sum_{n=0}^{\infty}  (-1)^n \frac{x^{2n}}{(2n)!}, \quad x \in \mathbb{R}\\
\ln(1+x) &= \sum_{n=0}^{\infty} \frac{(-1)^{n-1}}{n}x^n, \quad x \in \langle -1, 1 )\\
(1+x)^{\alpha} &= \sum_{n=0}^{\infty}\begin{pmatrix}a\\b  \end{pmatrix} x^n, \quad x \in \langle -1, 1 ), \alpha \in \mathbb{R}
\end{aligned}\]

V zádání je nějaká funkce, kterou pomocí těhle vzorečků, upravíme na nekonečnou řadu, hledá se potom poloměr konvergence R.

\[R= \frac{1}{q}\]
\[q = \lim_{x\to\infty} \sqrt[n]{f(x)} = \lim_{x\to\infty} \frac{f(x(n+1))}{f(x(n))}\]
Prostě je to něco jako podílové a odmocninové kritérium v MA1


\subsection*{Fourierova řada}
Tady se "jenom" dosazuje do vzorečků



\[a_0 = \frac{2}{T} \int_0^T f(x) dx\]
\[a_k = \frac{2}{T} \int_0^T f(x) \cos \left(\frac{2 \pi k}{T}x \right) dx\]
\[b_k = \frac{2}{T} \int_0^T f(x) \sin \left(\frac{2 \pi k}{T}x \right) dx\]
To se potom celý dosadí sem

\[\frac{a_0}{2} + \sum_{k=0}^{\infty}(a_k \cos(kx) + b_k \sin (kx))\]

Tady je nějaký seznam vztahů - není to tak těžký, jdou vymyslet na místě

$\newline$
$\cos(\pi k) = cos(-\pi k) = (-1)^n$ -  nejlepší je si to představit na jednotkové kružnici, v celých násobcích $\pi$ se střídá hodnota -1 a 1 - toho docílíme právě pomocí $(-1)^n$

$\newline$
$\sin(\pi k) = \sin(-\pi k) = 0$, sinus v celých násobcích $\pi$ nabývá nuly

$\newline$
$(-1)^n \cdot  (-1)^{n+1} = -1$, pro sudá n je $(-1)^n$ kladné a  $(-1)^{n+1}$ záporné a pro lichá n je to právě naopak - výsledek je -1

$\newline$
$-(-1)^n = (-1)^{n+1}$






\section*{Další příklady s postupem}

Zjistěte, zda je pole $\vec{F} = \left(\ln(1 + z), 2yz, \frac{x}{1 + z} + y^2\right)$ potenciální a v kladném případě nalezněte jeho potenciál.


\[\vec{\nabla} \times \vec{F} = \begin{vmatrix} e_1 & e_2 & e_3 \\
\frac{\partial F}{\partial x} & \frac{\partial F}{\partial y} & \frac{\partial F}{\partial z} \\ 
\ln(1 + z) & 2yz &  \frac{x}{1+z} + y^2 \end{vmatrix} = (2y - 2y)e_1 + \left(\frac{1}{1 + z} - \frac{1}{1 + z}\right)e_2 + (0 - 0)e_3 = 0 \]
Pole je potencionální

\[(1) \quad \frac{\partial F}{\partial x} = \ln(1+z) \]
\[(2) \quad  \frac{\partial F}{\partial y} = 2yz \]
\[(3) \quad  \frac{\partial F}{\partial z} = \frac{x}{1 + z} + y^2\]

$$
\begin{aligned}
1. \quad F(x, y, z) &= x \ln(1 + z) + c_1(y, z) \\
2. \quad \frac{\partial c_1}{\partial y} &= 2yz\\
3. \quad c_1(y,z) &= y^2z + c_2(z)\\
4. \quad F(x, y, z) &= x \ln(1 + z) + y^2z + c_2(z) \\
5. \quad \frac{x}{1 + z} + y^2 \frac{\partial c_2}{\partial z} &= \frac{x}{1 + z} + y^2\\
6. \quad c_2 &= 0 + c\\
7. \quad F(x, y, z) &= x\ln(1 + z) + y^2z + c
\end{aligned}
$$
Toto je výsledný potenciál




Přepište následující integrál
\[\int_1^2 \int_0^{2-x} f dy dx\]
nejprve v opačném pořádí integrace a pak v polárních souřadnicích se středem v počátku v pořadí ${d\rho d\varphi}$


\begin{center}
	\begin{tikzpicture}
		\draw[thick, ->](-0.1,0) -- (2,0) node[below] {$x$};
		\draw[thick,->](0,-0.3) node[align = left, below] {$O$} -- (0,2) node[left] {$y$};
		\draw[domain=-1:3] plot ({\x}, {2-\x});
		\draw[domain=-1:2, variable=\y] plot ({1}, {\y});
		\draw[domain=-1:2, variable=\y] plot ({2}, {\y});
	\end{tikzpicture}
\end{center}

\end{document}
